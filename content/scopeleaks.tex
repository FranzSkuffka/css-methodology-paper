\section{Scope Leaks}
\label{sec:scopeleaks}
CSS were created with the software design principle \gls{soc} and the intent to separate content and visual styling.\footnote{cf. \cite{hakonphdoncss} p.74}
In contrast to inline styles and embedded style sheets they are reusable across different \gls{html} documents.
The rules of a stylesheet referenced in an \gls{html}-document may apply to any part of the DOM if selectors match.
This is because the scope of \gls{css} rules is global by default.
We also call them `unscoped'.
When we consider a stylesheet as a program that is executed in the context of the \gls{dom}, the individual \gls{css}-rules behave like as impure functions executed in given order.
This means they alter their arguments.\footnote{cf. \cite{linearabstractmachine} p.158}

Similar to a program with impure functions, style sheets with of unscoped styles can have unexpected side-effects.
For example, when we add or modify a rule with the intention to modify the visual appearance of a specific element of the \gls{dom}, the style change can apply to other elements as well.
This so called `scope leak'\footnote{cf. \cite{mpgcss}} can occur in any \gls{html}-Document that references this very stylesheet.
When the developer undertaking the changes is not aware of all applications of the rule he changes, some of resulting effects are to be considered unforeseen.

The following drawing visualises the structrue of a hypothetical \gls{dom} :

\begin{figure}[H]
  \centering
  \Tree[.body 
        [.form
          [.label ]
          [.input ]
          [.button
            [.span.label ]
          ]
        ]
        [....
          [.div.team-member
            [.img ]
            [.label.name ]
          ]
        ]
      ]
\end{figure}

This rule was written with the intention to set the type of the label inside the form in bold weight:

\lstset{language=CSS3,caption={Rule with side effects}}
\begin{lstlisting}
label {
  font-weight: bold;
}
\end{lstlisting}

Applying these rules achieves the intended effect, but also changes the weight of the type the name of a team member is set in.
The \verb ...  denote an arbitrarily complex part of the \gls{dom}.
Depending on the complexity of the \gls{dom} and the tools at hand, the costs connected to locating the effects of a modification may be high.
The cascade and specificity mechanism of \gls{css} may add some complexity to side effects because individual rules have the potential to interact with each other.
We assume the presence of the following, more specific, unscoped rule:

\lstset{language=CSS3,caption={Rule with non-generic specificity}}
\begin{lstlisting}
.label{
  font-weight: light;
}
\end{lstlisting}

Now the first rule does not take effect because the class selector of the second rule is more specific.\footnote{cf. \cite{w3conselectors}}
To achieve the targeted visual appearance, further modifications are necessary, e.g. increasing the specificity of the new rule, which again results in an increased complexity of the source code.

The risk of unforeseen side effects in the development of generally unscoped \gls{css} is increased with the complexity of the codebase and the number of developers involved.
Scope leaks increase the cost with comprehending the source code, i.e. to acquire and maintain knowledge of the system that enables a developer to modify it without unexpected side-effects while achieving the desired effect.
We can draw a parallel between non-local rules and Wulf and Shaw's considerations of non-local variables as `a major contributing factor in programs which are difficult to understand' for similar reasons.\footnote{\cite{globalvariables} p.28}

It remains to mention that global \gls{css}-rules have a valid use case as base styles.
Base styles provide visual consistency across across multiple \gls{html} documents and use element selectors.\footnote{cf. \cite{mpgcss}}
