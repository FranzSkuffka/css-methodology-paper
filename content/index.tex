\chapter{Introduction}
\label{ch:intro}
The results of a poll on \gls{css} Preprocessors with 13 000 responses conducted in 2012 by online magazine CSS-Tricks display a significant preference of those who tried preprocessors towards them.\footnote{cf. \cite{preprocessorpoll}}
The popularity of preprocessors can be considered as a symptom of insufficiency \gls{css} as a tool for modern web projects.
Stylesheet size and structure do have little perceptible impact on both load time and rendering performance.\footnote{cf. \cite{soundersonselectorperformance}}\footnote{cf. \cite{atkinsondryscope}}
Browser inconsistencies have a relatively straightforward solution, namely normalize- or reset-stylesheets.\footnote{cf. \cite{meyeronreset}}\footnote{cf. \cite{gallagheronnormalize}}
Solutions to layout problems like collapsing margins or unexpected float behavior tend to be less straightforward, but are proven and undebatable.\footnote{cf. \cite{meyeronfloats}}\footnote{cf. \cite{meyeronmargins}}
However, artifacts of the search for effective and efficient methods to structure \gls{css} Code dates back to 2003.\footnote{cf. \cite{methodmails}}

In the recent years several different practices or sets of practices to structure stylesheets have been promoted by widely recognized companies and individuals of the web development community.
Among others, these include Yandex\footnote{cf. \cite{bem}}, Yahoo\footnote{cf. \cite{atomiccsssite}}.
In defiance of the abundance of methods and `best practices', resources that provide comparative information based on thorough examination are scarce.
We will approach common structural problems in the development of \gls{css} caused by the features, quirks and shortcomings of the language.
Consecutively, we will examine selected practices promoted as solutions for these problems.
We focus our analysis on CSS-only solutions but note relevant tools like preprocessors and template engines.
The aim of this paper is to provide comprehensible insight for frontend developers that want to challenge best practices.

The following conventions are used in this paper:
\verb ... signifies the presence of omitted code of arbitrary complexity.
File names like \verb \\ \verb index.html in code excerpts name the following lines of code.
Within an example the file \verb index.html references \verb style.css .

% CSS3-Standard

\chapter{Structural Problems of CSS}
In this chapter, we will identify structural problems of \gls{css} and explain negative effects with the help of code samples.
Consecutively we present solutions for those problems.
The structural problems are selected by the expected amount of technical debt their encounter adds to a project.
They are a sub-set of the problems thath the method sets mentioned in \autoref{ch:intro} are addressiing and belong in either of two categories:
Violations of the gls{dry} principle are a general problem in software development and inherent to \gls{css} because \gls{css} itself provides no means to reference values.
This includes selectors, attribute-value-pairs and larger structures like entires rules and parts of those.
But first, we will cover scope leaks as a cause for unexpected side-effects.

% Maintainability includes readability and scalability.
% A project is considered scalable here if increasing it's scale or complexity has little to no negative impacts on readability.
% validation of problem selection
% problem - rule grouping by semantic css WHAT DOES IT MEAN

\section{Scope Leaks}
\label{sec:scopeleaks}
CSS were created based on the design Principle \gls{soc} with the intent to separate content and visual style.
In contrast to inline styles and embedded stylesheets they are reusable across different \gls{html} documents.
The rules of a stylesheet referenced in an \gls{html}-document may apply to any part of the DOM if selectors match and the rule is not overwritten or shadowed.
Because the scope of \gls{css} rules is global, we can also describe them as `unscoped'.
When a stylesheet is viewed as a program that is executed in the context of the \gls{dom}, the individual \gls{css}-rules may be viewed as impure functions executed in given order, altering their arguments.\footnote{cf. \cite{linearabstractmachine} p.158}

Similar to a program with impure functions, the stylesheet with of unscoped styles can have unforeseen side effects.
This situation is also to consider when modifying such a stylesheet.
For example, when a change is made to a selector or the associated properties with the intention to modify the visual appearance of a specific element of the \gls{dom}, the rule created or modified can apply to other elements aswell.
This so called `scope leak'\footnote{cf. \cite{mpgcss}} can occur in any \gls{html}-Document that references this very stylesheet.
When the developer undertaking the changes is not aware of all applications of a specific rule, some of the visual changes are considered unforeseen.
We now illustrate the given situation, initially presenting a visualisation of a hypthetical \gls{dom}:

\begin{figure}[H]
  \centering
  \Tree[.body 
        [.form
          [.label ]
          [.input ]
          [.button
            [.span.label ]
          ]
        ]
        [....
          [.div.team-member
            [.img ]
            [.label.name ]
          ]
        ]
      ]
\end{figure}

The following rule was written with the intention to set the type of the label inside the form in bold weight:

\lstset{language=CSS3,caption={Rule with side effects}}
\begin{lstlisting}
label {
  font-weight: bold;
}
\end{lstlisting}

Applying these rules achieves the intended effect, but also changes the weight of the type the name of a team member is set in.
The \verb ... signify an arbitrarly complex \gls{dom}-structure.
While it is obvious to us here that the rule will set the name of an employee in bold type, it may not be in an actual project.
Depending on the here omitted complexity and the tools at hand, of the \gls{dom}, the effort connected to locating the effects of a modification may be large.
The cascade and specificity mechanism of \gls{css} adds some somplexity to possible side effects due to possible interactions between individual rules.
We assume that in any stylesheet that the document references occurs the following, more specific, unscoped rule:

\lstset{language=CSS3,caption={Rule with non-generic specificity}}
\begin{lstlisting}
.label{
  font-weight: light;
}
\end{lstlisting}

Now the \verb label -rule does not take effect, because the \verb .label -rule has a higher priority.\footnote{cf. \cite{w3conselectors}}
To achieve the desired results, we need to undertake further modifications, e.g. increasing the specificity of the new rule, which again results in an increased complexity of the source code.

The risk of unforeseen side effects in the development of generally unscoped \gls{css} is increased with the complexity of the codebase and the number of both active and developers.
Scope leaks increase the effort required to comprehend the source code, i.e. acquire and maintain knowledge of the system that enables a developer to modify it without unexpected side-effects while achieving the desired effect.
Here we can draw a parallel between non-local rules and Wulf and Shaw's considerations of non-local variables as `a major contributing factor in programs which are difficult to understand' for similar reasons.\footnote{\cite{globalvariables} p.28}
% dynamic web-applications

However, global or `base' styles are not inherently considered harmful.
Their usage is suitable to provide representational consistency inside single aswell across multiple \gls{html} documents.\footnote{cf. \cite{mpgcss}}

\section{Violations of the \acrlong{dry} Principle}
\label{sec:dryviolations}
The \gls{dry} Principle was enunciated in 2000 by Hunt and Thomas.
It states that 
`Every piece of knowledge must have a single, unambiguous, authoritative representation within a system.'\footnote{\cite{pragmaticprogrammer} p. 27}
Many violations of the DRY principle in CSS are directly linked to the lack of variables.
In the following listing we can observe what is most likely a violation of \gls{dry}

\lstset{language=CSS3,caption={Violation of DRY},label=listing:suitconvention}
\begin{lstlisting}
button {
  margin: 10px;
  ???
}

a {
  margin: 10px;
  ???
}
\end{lstlisting}

We assume that the intention of the developer was to give both links and buttons the same margin whose exact value is derived from a grid system.
The singular piece of knowledge originating in the grid system now has two authoritative representations in the code.
This lowers the maintainability because a single change of the grid dimensions results in two or more necessary of the code.
Even automated tools like find and replace are prone to errors of their operators as the operation may require detailed knowledge of the target system.\footnote{cf. \cite{humanautomation} p.408}
To exemplify this, we consider the following listing:
\lstset{language=CSS3,caption={Coincidental color definition overlap},label=listing:suitconvention}
\begin{lstlisting}
.important {
    background-color: red;
}

.error {
    background-color: red;
}
\end{lstlisting}
Here, by coincidence, elements decorated with the classes \verb error  and \verb important  are both styled with red background.
In common sense, information about an error is important, but not all important information is an error.
This means that the two occurences of \verb red  may change independently because they actually mean different things.
If this happens, a simple find and replace operation on all occurences of \verb red  with the intent to fulfill changed design requirements results in an unexpected side-effect.
The bigger a project gets, the more likely are such coincidental overlaps.


% Vendor-prefixes


\section{Solutions}
After describing \gls{css}-specific structural problems, we will now introduce methods take from the promoted set of practices.
We note that the methods presented may not be suitable for any developer or project due to both personal preference or established conventions and practices.
The methods covered here are introduced and evaluated based on code samples.
In the evaluation we will also consider semanticity of selectors.
This refers to what a reader can infer about the contents of a rule by reading the associated selector alone.
We note that semantics here are not to be confused with microformats and semantic \gls{html}5-Elements and thus are not relevant to \gls{seo}.

\subsection{Modularisation}

The css-guidelines \gls{bem} and \gls{suit} share the concept of blocks.
They define rules for structuring markup and class names, or rather rule selectors.
% \gls{suit} is inspired by and may be viewed as being evolved from \gls{bem}.
A block encapsulates \gls{html} and \gls{css} as well as `other implementation technologies'\footnote{\cite{bem}}  and are also referred to as components or modules.\footnote{cf. \cite{suit}}\footnote{cf. \cite{bem}} 
Both have very similar conventions for the construction of \gls{css} rule selectors and the related markup.
The following example displays the construction of component and child-element selectors in \gls{suit}.

% snake/dash/camel case

\lstset{language=CSS3,caption={SUIT naming convention},label=excerpt:suitconvention}
\begin{lstlisting}
// index.html

<div class='BlockName'>
  <div class='BlockName-descendantName>
  ...

// style.css

.BlockName {}
.BlockName-descendantName {}
\end{lstlisting}

\gls{bem} 
Whereas the delimiters can be chosen freely in \gls{bem} the naming conventions are more specific for \gls{suit}.
BEM recommends not nesting descendants of a block within each other.\footnote{cf. \cite{bemfaq}}
The reason for capitalising the root node selector is to avoid name collisions and thus expectedly better integration with existing code.\footnote{cf. \cite{bemvssuitquestion}}




% __________ scoping

To this date, while some browsers support scoped \verb <style> -tags, the CSS scoping module level 1 is still in draft.\footnote{cf. \cite{cssscopingmodule}}\footnote{cf. \cite{styletag}}
The \gls{w3c}-draft on CSS scoping defines a syntax that is similar to that of media queries.
However, media queries no not operate based on the \gls{dom}-structure but are scoping styles based on various properties of the rendering device.\footnote{cf. \cite{mediaqueries}}
Here it is possible to surround a set of rules with a selector that, if matched, limits these rules' scope to children of the matched element.

\begin{lstlisting}
@scope div {
  span {
  color: blue;
  }
}
\end{lstlisting}
In terms of limiting the visibility of rules this has the exact same effect as prefixing it with that selector.\footnote{cf. \cite{cssscopingmodule}}
Atkins described this syntax as `selector sugar'.\footnote{cf. \cite{atkinsonselectorsugar}}

A selector chain with more than one selector effectively scopes the rule by the first selector:

\lstset{language=CSS3,caption={Basic scoping},label=excerpt:scoping}
\begin{lstlisting}
.sidebar span {
  background: red;
}
.sidebar a {
  color: blue;
}
\end{lstlisting}

The conventions defined by SUIT and BEM achieve module-scoping differently.
\autoref{excerpt:suitconvention} shows that the scoping here does not utilise the cascade but works through prefixing the actual class names.
Block descendant styles are applied to elements decorated with the classname \verb ComponentName-descendantName , regardless of their position in the \gls{dom}.
Escalante criticises this approach.
Instead of leveraging the functionality of the cascade, \gls{bem} and \gls{suit} annul it to `replicate it [...] in a less efficient manner'.\footnote{\cite{mpgcss}}
Inheriting the class name of the block root into the class name of a descendant causes repetition in both \gls{css} and \gls{html}.\footnote{cf. \cite{mpgcss}}
{\slshape Title CSS} leverages the cascade to create a scope.\footnote{cf. \cite{titlecss}}
Through separating the block identifier as distinct class, repetition within the \gls{html}-markup is avoided:

\lstset{language=CSS3,caption={Title CSS Convention},label=excerpt:titlecss}
\begin{lstlisting}
.BlockName {}
.BlockName .descendantName{}
\end{lstlisting}

Besides avoiding repetition, Cuthbert adds ease-of-writing as well as readability as rationales for {\slshape Title CSS} as naming convention.
If Title CSS is executed in the style of the given scope leaks are still present.
Whereas BEM and SUIT target descendants directly through class names, the cascade may also target identically named descendants of blocks nested within other blocks.
To avoid this, we modify the conventions for creating descendant selectors like so: 

\lstset{language=CSS3,caption={Title CSS Convention modified},label=excerpt:titlecssfixed}
\begin{lstlisting}
.BlockName > .descendantName{}
\end{lstlisting}
The \verb > component of the selector ensures that only direct descendants of \verb BlockName are matched with \verb descendantName.
After presenting methods for avoiding scope leaks in \gls{css}, methods to avoid redundancy within \gls{css} files are presented in the following section.

% performance
% utility
% - overriding 
% shadowing






\subsection{Page Scoping}
Coyier introduced and Escalante recommends decorating the body element of each \gls{html} document of a project with an ID.\footnote{cf. \cite{coyieridbody}}\footnote{cf. \cite{mpgcss}}
Prefixing a selector with a specific page id scopes the associated rule to that page or page type.
Generally, page-specific styles may be used to define visual cues as visual orientation aids.
A concrete example is distinguishing the header of an article page from other pages.
% The major use case for page or page-type specific styles is the creation of visual exceptions among all pages.

\begin{lstlisting}
header {
  background: transparent;
}
#article header {
  background: black;
}
\end{lstlisting}

Coyier also exemplifies the use of page-specific styles through highlighting the currently selected entry of the main menu without changing the markup of the latter.

\begin{lstlisting}

// index.html

<body id='about'>
  <nav>
    <ul>
      <li class='nav-item home'>
      <li class='nav-item about'>
    ...


// style.css

nav li {}

#about .nav-item.about {}

\end{lstlisting}

However this technique should be applied with caution.
The markup and the styles in this example are coupled tightly and effectively violate the principle of \gls{soc} because there is no visual information about the currently selected entry without the stylesheet.
Each navigation entry requires an additional rule to function correctly in the context of the user interface.
Because this leads to bloat in \gls{css} code when the number of navigation entry is growing, a dynamic \gls{js} client- or server-side solution modifying the markup might be more when aiming for scalability.
Page-specific IDs are also considered `global modifiers'.
BEM does not accommodate the concept of global modifiers.\footnote{cf. \cite{bemfaq}}
The rationale behind this convention is not exposed, but it can be accounted to aforementioned caveats.
The following section explains how, among other methods, more granular modifiers can be utiliesed to reduce repetition

\subsection{Decoupling}
There are several tools and methods available that allow eliminating repetitions in \gls{css}-code.
Variables are, to this date, only fully supported by Firefox 31 and has been removed as an experimental feature in Google Chrome from version 33 to 34.\footnote{cf. \cite{cssvariables}}
To eliminate the duplication constants are sufficient.
Besides variables there are no native gls{css} means to emulate constants in their classical definition.

All common \gls{css}-Preprocessors provide a variable feature.
These variables are only variable during the actual processing and not at interpretation-time.
Thus preprocessor variables are effectively constants.\footnote{cf. \cite{wirthpreprocessors} p.27}
Using preprocessors allows eliminating repetition in the source code to some degree
but the repetition will still be observable in the processed \gls{css}.
Verou assumes that the usage of preprocessors may result in the developer `losing track of CSS filesize'.\footnote{cf. \cite{veroupreprocessors}}
This potential may not be as big when using preprocessed variables, but bigger for other language features like mixins.
Due to the lack of variables, 
% information size
% bandwidth for user

\subsubsection*{Atomic CSS}
An approach to completely eliminate repetition in native \gls{css} code is Atomic \gls{css}, presentend in 2013 by Koblentz, frontend developer at Yahoo.
It advocates splitting the \gls{css} source code in the smallest possible parts.
Atomic \gls{css} rules represent a specific visual appearance.
The class names of the individual rules are not semantic but representational.
This effectively decouples the \gls{css} from the structure of the \gls{html} document.

% approach by frameworks, high reusability

\lstset{language=CSS3,caption={Atomic CSS Rule}}
\begin{lstlisting}
.Mend-small {
    margin-left: 10px;
}
\end{lstlisting}

In contrast to block-based approaches like \gls{suit} and \gls{bem} Atomic CSS is no scoping and no nesting.
Rules do not have a specific context and thus are highly reusable.\footnote{cf. \cite{atomiccssarticle}}
There are multiple tradeoffs to the Atomic CSS approach:

\begin{itemize}
    \item {\normalfont \bfseries Additional classes in markup:} Visual attributes are grouped through class assignment rather than below a specific selector.
        This means more classes need to be assigned to each element whose visual appearence deviates from default and base styles.
        The resulting markup may be considered bloated.

    \item {\normalfont \bfseries Non-semantic class names:} Because of the radical decoupling the \gls{css} rules are named by the visual attribute they yield. 
        The lack of scoping possibly impedes the developer's ability to localize the effect of changes.

    \item {\normalfont \bfseries Changing names and unused rules:} The W3C states that `Good names don't change'.\footnote{\cite{classsemantics}}
        The Usage of class names that inherit attribute values are a violation of \gls{dry}. 
        Replacing these attribute values with names like \verb small may eleminate the need to change class names when the actual attribute values change.
\end{itemize}

The development of the Atomic CSS approach has since been continued by Yahoo but is not examined in detail, because the new methods involve the usage of processors.
Current versions do not involve direct editing of \gls{css}-files.
Class names define the actual style attributes.
Atomizer, a build tool that extracts \gls{css} rules from the definitions in the markup is used to generate gls{css}-files.
Atomizer supports the usage of variables.
Whereas an overhead of unused rules in the \gls{css}-files is prevented by this approach, the markup is still studded with non-semantically named classes.\footnote{cf. \cite{atomiccsssite}}
% vs inline styles
% applying and de-applying classes, meaning of classes, logic has to be stored in module context


\subsubsection*{Inheritance}
Whereas Atomic CSS recommends a quite radical approach, the methods described in \autoref{sec:modules} can be extended to emulate patterns from object oriented programming.

According to SOLID\footnote{Acrfull{SOLID}} CSS classes should be supposed to be modeled with a responsibility in mind, e.g. styling buttons.\footnote{\cite{solidcss}}
BEM and SUIT use this concept in the development of components.
Components can be extended by sub-classes.
In the following code excerpt the \verb BaseComponent styles are inherited through appending the selector of the extending class.
\lstset{language=CSS3,caption={Module extension}}
\begin{lstlisting}
.BaseComponent,
.BaseComponent-subclass {}

.BaseComponent-subclass {}

<div class='BaseComponent-subclass'>
\end{lstlisting}
The sub class may then be assigned to a DOM element to apply both base styles and extended styles to it.
This method respects the Open / Closed principle.
Base classes should have a single responsibility and thus be impartible.
If this is respected, a base class ideally does not change if the requirements to the UI are constant.
The base classes are then open to extension but closed to change.

Decorators have a similar effect.
Instead of modeling inheritance within the stylesheet, both base and decorator classes are modeled seperately.
The decorator is then applied to a DOM element with the according base class.
The visual apperance of this element is then modified with out altering that of the base class.
Decorators shift some repetition to the HTML document in contrast to subclassing:
\lstset{language=CSS3,caption={Module decorator}}
\begin{lstlisting}
.BaseComponent {}
.BaseComponent-decorator {}

<div class='BaseComponent BaseComponent-decorator'>
\end{lstlisting}


\subsubsection*{Utility classes}
Some styles are hard to link to specific classes exclusively.
This is reflected that it is usually hard to find a descendant name that makes sense the module in context.
A use case for utility classes is the implementation of workarounds for limitations of \gls{css}.
Like Atomic CSS classes, their names are based on their specific styles.
This makes them highly reusable.
The following example displays the use of a utility class to prevent single orphaned words which is not possible without additional markup:
\lstset{language=CSS3,caption={Utilities}}
\begin{lstlisting}
.u-nowrap {
    white-space: nowrap;
}

<h1>
    Lorem Ipsum dolor
    <span class='u-nowrap'> sit amet.</span>
</h1>
\end{lstlisting}
A class name like \verb .Heading-nowrap would break with the naming scheme of modules and limit the reusability of this class.
The heading is not any less of a heading of the \verb .u-nowrap element is missing.
This element is markup that was added with the sole purpose to extend the capabilities of HTML and CSS.
Also, in this example, if a template engine is in place, it makes sense to delegate the wrapping of the last two words to it.
Utility classes do not change.
They barely yield properties with numeric values but rather those with a limited set of options.


% utilities

\chapter{Conclusion}
\gls{css} as a language is easy to learn and hard to master.
The speed at which tangible results can be produced when using \gls{css} is fairly high, but may decline quickly.
Because of that, the development process of \gls{css} should implement adequate and effective methods to prevent scope leaks and gratuitous repetition with the aim to improve readability and scalability.
Cutting the technical debt caused by structural problems of \gls{css} involves not only modifying stylesheets but also the \gls{html}-documents that reference them.
The cost of this may be increased through possible tight coupling of other software components with the \gls{html}-component.

The examination of the selected methods showed that there are effective solutions available for preventing scope leaks and confining gratuitious repetition.
However, the methods aimed at enforcing \gls{dry} are mutually exclusive, as both naming conventions and mechanics.
They exist on a scale that ranges from complete decoupling and maximum reusability to flexible modularity with limited reusability.
In addition the efficiency of Atomic \gls{css} is limited by page scoping methods.
The decision about when to apply scoping methods and which naming or rather structure convention to select remains to be decided in concrete project context with respect to developer preferences and established conventions.

The list of problems and solutions evaluated in this paper is incomplete.
As the selection of problems and solution is based on estimations of negative impact on code structure and the attention attributed to each, it makes sense to examine other issues like state management, performance, browser inconsistencies etc.
A more complete view of language-specific problems in the development of CSS could enable researchers to design abstract pattern libraries that consider the most important issues without restrictions in implementation details.
However, in order to reach the level of knowledge similar to general knowledge in software engineering, more efforts must be made in the research of front-end development, specifically CSS.
