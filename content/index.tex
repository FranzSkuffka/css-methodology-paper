\chapter{Introduction}
\label{ch:intro}
The results of a poll on \gls{css} Preprocessors with 13 000 responses conducted in 2012 by online magazine CSS-Tricks display a significant preference of those who tried preprocessors towards them.\footnote{cf. \cite{preprocessorpoll}}
The popularity of preprocessors can be considered as a symptom of insufficiency \gls{css} as a tool for modern web projects.
Stylesheet size and structure do have little perceptible impact on both load time and rendering performance.\footnote{cf. \cite{soundersonselectorperformance}}\footnote{cf. \cite{atkinsondryscope}}
Browser inconsistencies have a relatively straightforward solution, namely normalize- or reset-stylesheets.\footnote{cf. \cite{meyeronreset}}\footnote{cf. \cite{gallagheronnormalize}}
Solutions to layout problems like collapsing margins or unexpected float behavior tend to be less straightforward, but are proven and undebatable.\footnote{cf. \cite{meyeronfloats}}\footnote{cf. \cite{meyeronmargins}}
However, artifacts of the search for effective and efficient methods to structure \gls{css} Code dates back to 2003.\footnote{cf. \cite{methodmails}}

In the recent years several different practices or sets of practices to structure stylesheets have been promoted by widely recognized companies and individuals of the web development community.
Among others, these include Yandex\footnote{cf. \cite{bem}}, Yahoo\footnote{cf. \cite{atomiccsssite}}.
In defiance of the abundance of methods and `best practices', resources that provide comparative information based on thorough examination are scarce.
We will approach common structural problems in the development of \gls{css} caused by the features, quirks and shortcomings of the language.
Consecutively, we will examine selected practices promoted as solutions for these problems.
We focus our analysis on CSS-only solutions but note relevant tools like preprocessors and template engines.
The aim of this paper is to provide comprehensible insight for frontend developers that want to challenge best practices.

The following conventions are used in this paper:
\verb ... signifies the presence of omitted code of arbitrary complexity.
File names like \verb \\ \verb index.html in code excerpts name the following lines of code.
Within an example the file \verb index.html references \verb style.css .

\chapter{Common Problems and Solutions}
fix solutions to problem
definition by patterns / anti patterns
'so benutzt man das'
- methodologies
- 
name und idee
historie
\section{Scope Leaks}
CSS were created by the design Principle \gls{soc} with the intent to separate content and visual style.
In contrast to inline styles and embedded stylesheets they are reusable across different \gls{html} documents.
To this date, while some browsers support scoped \verb <style> -tags, the CSS scoping module level 1 is still in draft.\footnote{cf. \cite{cssscopingmodule}}\footnote{cf. \cite{styletag}}
This means that the rules of a referenced stylesheet apply to any part of the DOM if selectors are matched and the rule is not overwritten or shadowed.
When a stylesheet is viewed as a program that is executed with the \gls{dom} as context, the \gls{css}-rules may be viewed as impure functions executed in given order, altering their arguments.\footnote{cf. \cite{linearabstractmachine} p.158}

Similar to a program with impure functions, the modification of a stylesheet with of unscoped styles can have unforeseen side effects.
For example, when a change is made to a selector or the associated properties with the intention to modify the visual appearance of a specific element of the \gls{dom}, the rule created or modified can apply to other elements aswell.
This so called `scope leak'\footnote{cf. \cite{mpgcss}} can occur in any \gls{html}-Document that references this very stylesheet.
When the developer undertaking the changes is not aware of all applications of a specific rule, some of the visual changes are considered unforeseen.
The cascade and Specificity mechanism of \gls{css} adds some somplexity to possible side effects due to possible interactions between individual rules.
The risk of unforeseen side effects in the development of generally unscoped \gls{css} is increased with the size of the codebase and the number of developers.
This is due to the increased amount of knowledge required to foresee side effects. 
Wulf and Shaw consider non-local variables `a major contributing factor in programs which are difficult to understand' for similar reasons.\footnote{\cite{globalvariables} p.28}

However, global or 'base' styles are not inherently considered harmful.
Their usage is suitable to provide representational consistency inside single aswell across multiple \gls{html} documents.\footnote{cf. \cite{mpgcss}}
Scoping makes sense to create exceptions aswell as styling more complex structures isolated structures within the DOM.
The effects of a change of scoped styles are easier to localize in comparison to unscoped styles.
So when the intended visual effect of a specific rule is not global, it makes sense to scope it in order to possibly reduce cognitive load for developers.

The aforementioned \gls{w3c}-draft on CSS scoping defines a syntax that is similar to that of media queries.
However, media queries no not operate based on the \gls{dom}-structure but are scoping styles based on various properties of the rendering device.\footnote{cf. \cite{mediaqueries}}
Here it is possible to surround a set of rules with a selector that, if matched, limits these rules' scope to children of the matched element.
\begin{verbatim}
    @scope div {
      span {
        color: blue;
      }
    }
\end{verbatim}
In terms of limiting the visibility of rules this has the exact same effect as prefixing it with that selector.\footnote{cf. \cite{cssscopingmodule}}

A selector chain with more than one selector scopes the rule by the first selector:
\begin{verbatim}
.sidebar span {
    background: red;
}
.sidebar a {
    color: blue;
}
\end{verbatim}
\begin{figure}
\caption{The rules are scoped within the context of sidebar elements}
\end{figure}

The usage of this technique is often observed to limit the scope of styles to a specific page.
Coyier introduced and Escalante recommends decorating decorating the body tag of each \gls{html} document of a project with an ID.\footnote{cf. \cite{coyieridbody}}\footnote{cf. \cite{mpgcss}}
The major use case for page or page-type specific styles is the creation of visual exceptions among all pages.
Generally, page-specific styles may be used to define visual cues as visual orientation aids.
A concrete example for this is a visual design specification that distinguishes the header of an article page through a an opaque background instead of being  a transparent layer over a darker image.

\begin{verbatim}
header {
    background: transparent;
}
#article header {
    background: black;
}
\end{verbatim}

Coyier also exemplifies the use of page-specific styles through highlighting the currently selected entry of the main menu without changing the markup of the latter.

\begin{verbatim}

// index.html

<body id='about'>
    <nav>
        <ul>
            <li class='nav-item home'>
            <li class='nav-item about'>
        ...


// style.css

nav li {}

#about .nav-item.about {}

\end{verbatim}

However this technique should be applied with caution.
The markup and the styles in this example are coupled tightly and effectively violate the principle of \gls{soc}.
Each navigation entry requires an additional rule to function correctly in the context of the user interface.
Because this leads to bloat in \gls{css} code when the number of navigation entry is growing, a dynamic \gls{js} client- or server-side solution modifying the markup might be more when aiming for scalability.

The methodologies \gls{bem} and \gls{suit} share the concept of creating indivisble blocks in both .
Blocks encapsulate \gls{html} and \gls{css} as well as `other implementation technologies'\footnote{\cite{bem}}  and are also referred to as `components'.\footnote{cf. \cite{suit}}\footnote{cf. \cite{bem}} 
\gls{suit} is inspired by and may be viewed as being evolved from \gls{suit}.
Both have very similar conventions for the construction of \gls{css} rule selectors and the related markup.
The following example displays the construction of component and child-element selectors in \gls{suit}.
snake/dash/camel case

\begin{verbatim}
.BlockName {}

.BlockName-descendantName {}
\end{verbatim}
Whereas the delimiters can be chosen freely in \gls{bem} the naming conventions are more specific for \gls{suit}.
The reason for capitalising the root node selector is to avoid name collisions and thus expectedly better integration with existing code.\footnote{cf. \cite{bemvssuitquestion}}
The example shows that the scoping here does not utilise the cascade but works through prefixing the actual class names.
Descendent styles are applied to elements decorated with the classname \verb ComponentName-descendantName , regardless of their position in the \gls{dom}.

Escalante criticises this approach.
Instead of leveraging the functionality of the cascade, \gls{bem} and \gls{suit} annul it to `replicate it [...] in a less efficient manner'.\footnote{\cite{mpgcss}}
Inheriting the class name of the block root into the class name of a descendant causes repetitition and thus redundancy in the \gls{html}.\footnote{cf. \cite{mpgcss}}
{\slshape Title CSS} leverages the cascade to create a scope.
Through separating the block identifier as distinct class, repetition within the \gls{html}-markup is avoided:

\begin{verbatim}
.BlockName {}

.BlockName .descendantName{}
\end{verbatim}
Besides avoiding repetition, Cuthbert adds ease-of-writing as well as readability as rationales for {\slshape Title CSS} as naming convention.
After presenting methods for avoiding scope leaks in \gls{css}, methods to avoid redundancy within \gls{css} files are presented in the following section.

% performance
% utility
% - overriding 
% shadowing


\section{Violations of the \acrlong{dry} Principle}
The \gls{dry} Principle was formulated in 2000 by Hunt and Thomas.
It states that 
`Every piece of knowledge must have a single, unambiguous, authoritative representation within a system.'\footnote{\cite{pragmaticprogrammer} p. 27}
When we consider the following excerpt we can observe what is most likely a violation of \gls{dry}

\begin{verbatim}
button {
    color: red;
    padding: 5px;
}

a {
    color: red;
    text-decoration: none;
}
\end{verbatim}

Assuming that the intention of the developer was to give both links and buttons an identical color, this piece of knowledge has two authoritative representations here.
But it is possible that the red color may only be identical by coincidence, meaning that it different semantics within the context of either element.
This implies that both definitions will not necessarily be change at the same time.
However, if this is not the case, that repetition may cause additional cost during maintenance.
This is because, if the inherently singular piece of knowledge changes, the code has to be changed in more that one place.
Even automated tools like find and replace are prone to errors of their operators as the operation may require detailed knowledge of the target system.\footnote{cf. \cite{humanautomation} p.408}

There are multiple options to eliminating the problem that has just been examined.
Variables are, to this date, only fully supported by Firefox 31 and has been removed as an experimental feature in Google Chrome from version 33 to 34.\footnote{cf. \cite{cssvariables}}
To eliminate the duplication constants are sufficient.
Besides variables there are no native gls{css} means to emulate constants in their classical definition.

All common \gls{css}-Preprocessors provide a variable feature.
These variables are only variable during the actual processing and not at interpretation-time.
Thus preprocessor variables are effectively constants.\footnote{cf. \cite{wirthpreprocessors} p.27}

input systems and output system
information size
bandwidth for user



no variables
repeated within the markup

respect to SOLID design principles\footnote{cf. \cite{solidcss}}
sub-classes = re-usability through dry
cascade

DRY Principle \footnote{cf. \cite{pragmaticprogrammer}}
low reusability
- specificity
- tight coupling with markup

Modules / Components
modifiers on components or elements

high reusability

balance between repeating 

extreme decoupling and reusability: Atomic design
downside: non-semantic
good names don't change
css-frameworks

DRY CSS: breaks code structure

\section{Tight Coupling with Markup}
Some solutions apply or do not apply to specific programmaning styles and workflows
% There are multiple difference between the scoping module and prefixing rule selectors with another selector.
% However the biggest difference is absolute specifity, i.e. the cascade prioritize scoped styles.
% The following section introduces multiple methods to scope \gls{css} rules.
% All these methods work on information provided by the \gls{dom}.
% There are two observed established techniques to utilise rules scoped with prefixes.
% The most frequently observed usage of 
